% @Author: Jérémy Coatelen
% @Date:   2015-05-05 19:28:31
% @Last Modified by:   dash
% @Last Modified time: 2015-05-05 20:56:31

% Classe de base de type 'report'
\documentclass{report}

% Pour que LaTeX comprenne les caractères accentués ;
% norme iso-8859, cela risque de ne pas marcher avec
% des fichiers créés sous windows
\usepackage[T1]{fontenc} 	

% Unicode
\usepackage[utf8x]{inputenc}	

% Pour adopter les règles de typographie française
\usepackage[frenchb]{babel} 	

% Fonte plus grasse 
\usepackage{lmodern}		            

% Mise en page
\usepackage{geometry}		
\geometry{a4paper}

% Contrôle du header et du footer
\usepackage{fancyhdr}		
\pagestyle{fancy}

% Utilisation de la couleur 
\usepackage{xcolor} 	

% Utilisation des images	                    
\usepackage{graphicx} 		

% Utilisation des sommaires dans les sections
\usepackage{minitoc}		
\mtcselectlanguage{french} 
\def\mtctitle{Contenu}

% Insertion d'algorithmes
\usepackage[chapter]{algorithm} 
\usepackage{algorithmic}

% Ecriture des codes sources
\usepackage{listings}		

% Utilisation des hyperliens
\usepackage[hyperfootnotes=false]{hyperref}	

% Liens corrigés vers les figures etc via la capture
\usepackage[all]{hypcap}	
				
% Utilisation des sous-figures
\usepackage{caption}
\usepackage{subcaption}

% Utilisation des annexes
\usepackage{appendix}		

% Utilisation des rétroréférencements sur les citations
\usepackage[numbers,sort&compress]{natbib}
%\usepackage{chapterbib}		% en mode section pour avoir une biblio par chapitre

% Insertion d'un glossaire
\usepackage{glossaries}		

% Insertion de pdf externes comme nvlls pages
\usepackage{pdfpages}		

% Numérotation du sommaire
\setcounter{tocdepth}{4}
%\usepackage{tocloft}
%\renewcommand\cftchapfont{\LARGE\bfseries}
%\renewcommand\cftsecfont{\small}
%\renewcommand\cftchappagefont{\LARGE\bfseries}
%\renewcommand\cftsecpagefont{\LARGE}

% Numérotation des paragraphes
\setcounter{secnumdepth}{5}

% texte justifié
\usepackage{ragged2e} 	

% Génération de texte pour le template	
\usepackage{lipsum}

% Todo notes
\usepackage[textsize=small, colorinlistoftodos]{todonotes}

% Interligne
\usepackage{setspace}		
\onehalfspacing

% utilisation de \ifstrequal
\usepackage{pdftexcmds}		

% fusion des lignes d'un tableau
\usepackage{multirow}		

% (a), (b), ... dans un enumerate
\usepackage{enumitem}		

% Ajout des numéros de lignes 
% http://texblog.org/2012/02/08/adding-line-numbers-to-documents/
%\usepackage[pagewise, modulo]{lineno}

% nouveaux caractères comme \textupfullarrow
\usepackage{tipx}		

% notamment pour les block de texte encadrés
\usepackage{fancybox}		
    
% Liste des équations
\usepackage{tocloft}

% Les bibliothèques LaTeX de l'American
% Mathematical Society
\usepackage{amsmath} 		
\usepackage{amssymb}

% accolades dans les fonctions
\usepackage{cases}	

% expression mathématiques en gras	
\usepackage{bm}			

% Crochets entiers naturels
\usepackage{stmaryrd}	

% Polices maths
\usepackage{mathrsfs}

% unités internationales
\usepackage{siunitx}

% Tabular extensibles
\newcommand{\head}[1]{\textnormal{\textbf{#1}}}
\definecolor{clr-lightgray}{gray}{0.95}
\usepackage{tabularx}	
\newcolumntype{P}[1]{>{\RaggedRight\hspace{0pt}}p{#1}}


% Définition du dossier des images
\graphicspath{{./images/}}

% Parametres des hyperliens
\definecolor{coulLiens}{rgb}{0.05,0.19,0.52}
\definecolor{coulCitations}{rgb}{0.0,0.0,0.5}
\hypersetup{
     backref=true,    %permet d'ajouter des liens dans...
     pagebackref=true,%...les bibliographies
     hyperindex=true, %ajoute des liens dans les index.
     colorlinks=true, %colorise les liens
     breaklinks=true, %permet le retour à la ligne dans les liens trop longs
     urlcolor= blue, %couleur des 
     citecolor=coulCitations,
     linkcolor= coulLiens, %couleur des liens internes
     bookmarks=true, %créé des signets pour Acrobat
     bookmarksopen=true,            %si les signets Acrobat sont créés,
                                    %les afficher complètement.
     pdftitle={M\'emoire de th\`ese}, %informations apparaissant dans
     pdfauthor={COATELEN J\'er\'emy},     %dans les informations du document
     pdfsubject={}          %sous Acrobat.
}

% Redéfinition des termes utilisés dans les algorithmes
\renewcommand{\algorithmicrequire} {\textbf{\textsc{Entrées:}}}
\renewcommand{\algorithmicensure}  {\textbf{\textsc{Sorties:}}}
\renewcommand{\algorithmicwhile}   {\textbf{tantque}}
\renewcommand{\algorithmicdo}      {\textbf{faire}}
\renewcommand{\algorithmicendwhile}{\textbf{fin tantque}}
\renewcommand{\algorithmicend}     {\textbf{fin}}
\renewcommand{\algorithmicif}      {\textbf{si}}
\renewcommand{\algorithmicendif}   {\textbf{fin si}}
\renewcommand{\algorithmicelse}    {\textbf{sinon}}
\renewcommand{\algorithmicthen}    {\textbf{alors}}
\renewcommand{\algorithmicfor}     {\textbf{pour}}
\renewcommand{\algorithmicforall}  {\textbf{pour tout}}
\renewcommand{\algorithmicdo}      {\textbf{faire}}
\renewcommand{\algorithmicendfor}  {\textbf{fin pour}}
\renewcommand{\algorithmicloop}    {\textbf{boucler}}
\renewcommand{\algorithmicendloop} {\textbf{fin boucle}}
\renewcommand{\algorithmicrepeat}  {\textbf{répéter}}
\renewcommand{\algorithmicuntil}   {\textbf{jusqu'à}}
\renewcommand{\algorithmicreturn}  {\textbf{retourner}}
\floatname{algorithm}{Algorithme}
\renewcommand{\listalgorithmname}	 {Table des Algorithmes}
\newcommand{\theHalgorithm}{\arabic{algorithm}}

% Macros générales
\newcommand{\ea}{\emph{et al.}}
\newcommand{\cad}{c'est\tiret à\tiret dire}
\newcommand{\vav}{vis\tiret à\tiret vis}

% Redéfinition des entêtes et pieds de page
\renewcommand{\chaptermark}[1]{\markboth{\bsc{\chaptername~\thechapter{} :} #1}{}}
\renewcommand{\sectionmark}[1]{\markright{\thesection{} #1}}\fancyhead{}
\fancyfoot{}
\lhead{\textsl{\rightmark}}
\chead{}
\rhead{\textsl{\leftmark}}
\lfoot{Prénom Nom}
\cfoot{\textbf{\thepage}~~\hyperref[TOC]{\textupfullarrow TdM\textupfullarrow}}
\rfoot{Mémoire de thèse}
\renewcommand{\headrulewidth}{0.4pt}
\renewcommand{\footrulewidth}{0.4pt}

% Ligne horizontale
\newcommand*{\ligne}[1][1]{\begin{center}\rule[0.4em]{0.5\textwidth}{0.5pt}\par\end{center}}

% Todos personnalisés
\makeatletter
\newcommand{\todocol}[1]{%
  \ifnum\pdf@strcmp{#1}{reference}=0
    green!40%
  \else\ifnum\pdf@strcmp{#1}{explain}=0
    red!40%
  \else\ifnum\pdf@strcmp{#1}{replace}=0
    blue!40%
  \else\ifnum\pdf@strcmp{#1}{figure}=0
    pink!40%
  \else
    unknown%
  \fi\fi\fi\fi}
\makeatother

% Parametres des codes sources
\definecolor{colKeys}{rgb}{0,0,1}
\definecolor{colIdentifier}{rgb}{0,0,0}
\definecolor{colComments}{rgb}{0,0.5,1}
\definecolor{colString}{rgb}{0.6,0.1,0.1}
\lstset{%
  float=hbp,%
  basicstyle=\ttfamily\small, %
  identifierstyle=\color{colIdentifier}, %
  keywordstyle=\color{colKeys}, %
  stringstyle=\color{colString}, %
  commentstyle=\color{colComments}, %
  columns=flexible, %
  tabsize=4, %2
  frame=trBL, %
  frameround=tttt, %
  extendedchars=true, %
  showspaces=false, %
  showstringspaces=false, %
  numbers=left, %
  numberstyle=\tiny, %
  breaklines=true, %
  breakautoindent=true, %
  captionpos=b,%
  xrightmargin=0cm, %
  xleftmargin=0cm
} 

% Commandes maths
\newtheorem{de}{Définition}[subsection] % les définitions et les théorèmes sont
\newtheorem{theo}{Théorème}[section]    % numérotés par section
\newtheorem{prop}[theo]{Proposition}    % Les propositions ont le même compteur
                                        % que les théorèmes

\newcommand{\ordmult}{\mathop{.}} % \ordmult est l'opérateur de multiplication
                                  % d'ordinaux
\newcommand{\expo}[2]{#1^{#2}}
\renewcommand{\degre}{^{\circ}}
\DeclareMathOperator*{\argmin}{argmin}
\DeclareMathOperator*{\argmax}{argmax}

% Minitoc customisé
\newcommand{\myminitoc}
{
	\minitoc
	\vspace{0.5cm}
}

% * Hack du fichier BST par défaut * 
% Lien vers style personnalisé des citations
% et Commande pour la double compatibilité des citations fr/en
\newcommand{\mybibliostyle}{myunsrtnat-fr}
\DeclareRobustCommand\authorand{et}

% * Hack pour régler les problèmes césure liées à l'utilisation
% du caractère '-' seul *
% nouveau trai d'union pour la correction des césures
\newcommand\tiret{\nobreak\hskip0pt-\nobreak\hskip0pt\relax}

% Commandes pour natbib
\newcommand{\citeauthorandyear}[1]{(\citeauthor{#1}, \citeyear{#1})}


% Défintion des titres à cacher dans le TOC
\newcommand{\hiddensubsection}[1]{
\stepcounter{subsection}
\subsection*{\arabic{chapter}.\arabic{section}.\arabic{subsection}%
\hspace{1em}{#1}}
}
\newcommand{\hiddensubsubsection}[1]{
\stepcounter{subsubsection}
\subsubsection*{\arabic{chapter}.\arabic{section}.\arabic{subsection}%
.\arabic{subsubsection}\hspace{1em}{#1}}
}
\newcommand{\hiddenparagraph}[1]{
  \stepcounter{paragraph}
  \paragraph*{#1}
}
\newcommand{\hiddensubparagraph}[1]{
  \stepcounter{subparagraph}
  \subparagraph*{#1}
}


% * Hack pour utiliser une version courte d'un titre dans l'entête *
\newcommand{\markedsection}[2]{\section[#2]{#2%
\sectionmark{#1}}
\sectionmark{#1}}
\newcommand{\markedsubsection}[2]{\subsection[#2]{#2%
\subsectionmark{#1}}
\subsectionmark{#1}}

% guillemets françaises
\newcommand{\gui}[1]{\og #1 \fg{}}

% Citer de l'anglais 
\newcommand{\anglais}[1]{\gui{\textit{#1}}}

% Commande pour ajouter une équation à la table des équations
\newcommand{\listequationsname}{Table des équations}
\newlistof{myequations}{equ}{\listequationsname}
\newcommand{\myequations}[1]
{\addcontentsline{equ}{myequations}{\protect\numberline{\theequation}~~~#1}}

% Pour l'écartement des lignes de tableau 
\newcommand{\stretchthistab}[1]{\renewcommand{\arraystretch}{#1}}




% Définition du glassaire
\include{glossaire}
\makeglossaries	




%%%%%%%%%%%%%%%%%%%%%%%%%%%%%%%%%%%%%%%%%%%%%%%%%%%%%%%%%%%%
\begin{document}

% Initialisation des minitoc
\dominitoc 

% Parametres des codes sources
\lstset{language=c++}
\lstset{commentstyle=\textit}
\setcounter{secnumdepth}{3}

% Numérotation des pages en chiffres romains pour les 
% premières pages du mémoire
\pagenumbering{roman}

% Interligne simple pour ces premières pages
\singlespacing

%%%%%%%%%%%%%%%%%%%%%%%%%%%%%%%%%%%%%%%%%%%%%%%%%%%%%%%%%%%%
% La premières page est réservées à la liste des todos
\thispagestyle{empty}
\listoftodos
\null
\pagebreak

% Insertion de la page de garde
% @Author: Jérémy Coatelen
% @Date:   2015-05-05 20:28:57
% @Last Modified by:   dash00
% @Last Modified time: 2015-05-05 21:05:43

\thispagestyle{empty}

%\includegraphics[scale=0.05]{im2}

{\large

\vspace*{1cm}

\begin{center}

{\bf TH\`ESE DE DOCTORAT DE \\ L'UNIVERSIT\'E DE XXXXXXX}

\vspace*{0.5cm}

Sp\'ecialit\'e \\ [2ex]
{\bf Indiquer ici la spécialité}\ \\ 

\vspace*{0.5cm}

École doctorale XXXXXXX (CITY)

\vspace*{1cm}


Pr\'esent\'ee par \ \\


\vspace*{0.5cm}


{\Large {\bf Prénom Nom}}

\vspace*{1cm}
Pour obtenir le grade de \ \\[1ex]
{\bf DOCTEUR de l'UNIVERSIT\'E DE XXXXXXX} \ \\

\vspace*{1cm}

\end{center}

\flushleft{Sujet de la th\`ese :\ \\
\ \\
{\centering \Large {\bf TITRE DE LA THESE \\ }}
  

\vspace*{1cm}
\flushleft{soutenue le X xxxx 2015}\\[1ex]
\flushleft{devant le jury composé de :  }\\[1ex]
\flushleft{\begin{tabular}{r@{\ }ll}
  & Pr XXXXXX {\sc XXX} & Directeur de thèse\\
  & M. Prénom {\sc Nom} & Rapporteur \\
  & M. Prénom {\sc Nom} & Rapporteur  \\
  & M. Prénom {\sc Nom} & Examinateur  \\
  & M. Prénom {\sc Nom} & Examinateur  \\
\end{tabular}}

}



\justifying

%%%%%%%%%%%%%%%%%%%%%%%%%%%%%%%%%%%%%%%%%%%%%%%%%%%%%%%%%%%%
% Page du résumé
% @Author: Jérémy Coatelen
% @Date:   2015-05-05 20:21:23
% @Last Modified by:   dash00
% @Last Modified time: 2015-05-05 21:05:05

\clearpage
\phantomsection
%\addstarredchapter{Résumé}

\markboth{Abstract}{Résumé}

\vspace{4cm}

\begin{center}
\begin{large}\textbf{Titre}\end{large}\\
\end{center}
\vspace{1cm}

\begin{center}
\begin{large}\textbf{Résumé}\end{large}\\
\end{center}
\vspace{1cm}


Mots-clés : 


\vspace{2.5cm}
\ligne
\vspace{2.5cm}

\begin{center}
\begin{large}\textbf{Title}\end{large}\\
\end{center}
\vspace{1cm}



\vspace{1cm}
\begin{center}
\begin{large}\textbf{Abstract}\end{large}\\
\end{center}
\vspace{1cm}


Keywords : 


%%%%%%%%%%%%%%%%%%%%%%%%%%%%%%%%%%%%%%%%%%%%%%%%%%%%%%%%%%%%
% Page des adresses des collaborateurs 
% @Author: Jérémy Coatelen
% @Date:   2015-05-05 20:26:52
% @Last Modified by:   dash00
% @Last Modified time: 2015-05-05 21:01:56

\clearpage
\phantomsection
\chapter*{Adresses}

\begin{tabular}{lr}
\multicolumn{2}{l}{
 \begin{minipage}{0.8\linewidth}
  \label{adr:1}\textbf{Un nom très très long, mais vraiment très long ... pas encore assez long ... oui là c'est assez long !}\\
  Adresse de l'établissement\\
  Pays 
\end{minipage}}
\\[10ex]
\begin{minipage}{0.6\linewidth}
  \label{adr:2}\textbf{Un collaborateur}\\
  Adresse de l'établissement\\
  Pays 
\end{minipage} & 
\begin{minipage}{0.3\linewidth}
\includegraphics[width=\textwidth,height=\textheight,keepaspectratio]
{missing} 
\end{minipage}
\\[10ex]
\begin{minipage}{0.6\linewidth}
  \label{adr:3}\textbf{Un autre collaborateur}\\
  Adresse de l'établissement\\
  Pays 
\end{minipage} & 
\begin{minipage}{0.3\linewidth}
\includegraphics[width=\textwidth,height=\textheight,keepaspectratio]
{missing} 
\end{minipage}
\\[10ex]
\begin{minipage}{0.6\linewidth}
  \label{adr:4}\textbf{Un autre collaborateur}\\
  Adresse de l'établissement\\
  Pays 
\end{minipage} & 
\begin{minipage}{0.3\linewidth}
\includegraphics[width=\textwidth,height=\textheight,keepaspectratio]
{missing} 
\end{minipage}
\end{tabular}

%%%%%%%%%%%%%%%%%%%%%%%%%%%%%%%%%%%%%%%%%%%%%%%%%%%%%%%%%%%%
% Page des remerciements
% @Author: Jérémy Coatelen
% @Date:   2015-05-05 20:21:23
% @Last Modified by:   dash00
% @Last Modified time: 2015-05-05 21:04:54

\clearpage
\phantomsection
%\addstarredchapter{Remerciements}
\chapter*{Remerciements}

%%%%%%%%%%%%%%%%%%%%%%%%%%%%%%%%%%%%%%%%%%%%%%%%%%%%%%%%%%%%
% Pages du glossaire
\clearpage
\phantomsection
\addstarredchapter{Glossaire}
\printglossaries

%%%%%%%%%%%%%%%%%%%%%%%%%%%%%%%%%%%%%%%%%%%%%%%%%%%%%%%%%%%%
% Liste des symboles
% @Author: Jérémy Coatelen
% @Date:   2015-05-05 20:30:33
% @Last Modified by:   dash00
% @Last Modified time: 2015-05-05 21:03:44

\clearpage
\phantomsection
\addstarredchapter{Table des symboles}
\chapter*{Table des symboles}

{
\centering
\renewcommand{\arraystretch}{2.0}
\begin{tabularx}{\textwidth}{lX}
\textbf{$I$} & Désigne une image ou l'ensemble des pixels d'une image, pour
simplifier les notations on confondra image $I$, ensemble image $x \mapsto
I(x)$ et ensemble de pixels $x \in I$. \\
\textbf{$W$} & Largeur de l'image (en pixels) \\
\textbf{$H$} & Hauteur de l'image (en pixels) 
\end{tabularx}
}


%%%%%%%%%%%%%%%%%%%%%%%%%%%%%%%%%%%%%%%%%%%%%%%%%%%%%%%%%%%% 
% PAges du sommaire
\tableofcontents\label{TOC}

% Mise à jour de l'interligne
\onehalfspacing

% Numérotation des pages en chiffres standards
\clearpage
\pagenumbering{arabic}

%\include{avantpropos}

%\linenumbers

%%%%%%%%%%%%%%%%%%%%%%%%%%%%%%%%%%%%%%%%%%%%%%%%%%%%%%%%%%%%
% Insertion de l'introduction
% @Author: Jérémy Coatelen
% @Date:   2015-05-05 20:21:23
% @Last Modified by:   dash
% @Last Modified time: 2015-05-05 21:13:49

\chapter{Introduction générale}
\label{chap:introduction}

\myminitoc


% ===========================================================================
% ===========================================================================
\section{Présentation des options}

% ===========================================================================
\subsection{Fonctions de texte}

\subsubsection{Insertion de citations}

Utiliser des citations : contractée \citep{small} ou avec les auteurs \citet{big}. 
La langue par défaut est le français mais si vous désirez inclure de l'anglais et
toute en bénéficiant toujours de natbib alors utiliser ce hack (remplacement du 
mot clé 'et' par 'and') : 
\\

\DeclareRobustCommand\authorand{and}
This sentence is in english but no worries, natbib will use 'and' : 
\citet{twoauthors}
\DeclareRobustCommand\authorand{et}


\subsubsection{Insertion d'une figure}

Insertion d'une simple figure (figure~\ref{fig:thelabel}).

\begin{figure}[!t]
  \centering
  \includegraphics[width=0.6\linewidth]{missing}
  \caption{\label{fig:thelabel} One figure.}
\end{figure} 


\subsubsection{Insertion de todo notes}

Une todo note ici\todo{une todo note simple} ou\todo[color=\todocol{explain}]{une 
autre mais colorée} une todo note inline~: 
\todo[inline]{une todo note inline}

% ===========================================================================
\subsection{Fonctions techniques}

Insertion de l'équation~\ref{eqn:anequation}~:
\begin{equation}
\label{eqn:anequation}
x = 5
\myequations{My equation}
\end{equation}


Insertion de l'algorithme~\ref{algo:analgo}~:
\begin{algorithm}[H]
\caption{\label{algo:analgo}Un algorithme.}
\begin{algorithmic}[1]
\REQUIRE $n \geq 0 \vee x \neq 0$
    \ENSURE $y = x^n$
    \STATE $y \Leftarrow 1$
    \IF{$n < 0$}
        \STATE $X \Leftarrow 1 / x$
        \STATE $N \Leftarrow -n$
    \ELSE
        \STATE $X \Leftarrow x$
        \STATE $N \Leftarrow n$
    \ENDIF
    \WHILE{$N \neq 0$}
        \IF{$N$ is even}
            \STATE $X \Leftarrow X \times X$
            \STATE $N \Leftarrow N / 2$
        \ELSE[$N$ is odd]
            \STATE $y \Leftarrow y \times X$
            \STATE $N \Leftarrow N - 1$
        \ENDIF
    \ENDWHILE
\end{algorithmic}
\end{algorithm}

% ===========================================================================
% ===========================================================================
\section{Une section}


% ===========================================================================
\subsection{Une sous\tiret section}

\subsubsection{Une sous\tiret sous\tiret section}

\hiddenparagraph{Un paragraphe caché dans les TOC.}


\lipsum[4-8]



% ===========================================================================
\subsection{Une autre sous\tiret section}



%%%%%%%%%%%%%%%%%%%%%%%%%%%%%%%%%%%%%%%%%%%%%%%%%%%%%%%%%%%%
% Insertion du chapitre suivant
% @Author: Jérémy Coatelen
% @Date:   2015-05-05 20:42:25
% @Last Modified by:   dash00
% @Last Modified time: 2015-05-05 21:02:19

\chapter[Version courte d'un titre long]
{Titre de chapitre en version longue, très longue qui prend beaucoup de place mais qu'on sait gérer pour les TOCS}
\chaptermark{Version abrégée titre chapitre \dots}
\label{chap:achapter}

\myminitoc

% ===========================================================================
\section{A section}

\subsection{A subsection}

\subsection{Another subsection}
\lipsum

%%%%%%%%%%%%%%%%%%%%%%%%%%%%%%%%%%%%%%%%%%%%%%%%%%%%%%%%%%%%
% Insertion du chapitre suivant
% @Author: Jérémy Coatelen
% @Date:   2015-05-05 20:21:23
% @Last Modified by:   dash00
% @Last Modified time: 2015-05-05 21:02:29

\chapter{Troisième chapitre}
\label{chap:anotherchapter}

\myminitoc

% ===========================================================================
\section{A section}

\subsection{A subsection}

\subsection{Another subsection}


%%%%%%%%%%%%%%%%%%%%%%%%%%%%%%%%%%%%%%%%%%%%%%%%%%%%%%%%%%%%
% Insertion de la conclusion
% @Author: Jérémy Coatelen
% @Date:   2015-05-05 20:21:23
% @Last Modified by:   dash00
% @Last Modified time: 2015-05-05 21:02:45

\chapter{Conclusion}
\label{chap:conclusion}



% Mise à jour de l'interligne
\singlespacing

%%%%%%%%%%%%%%%%%%%%%%%%%%%%%%%%%%%%%%%%%%%%%%%%%%%%%%%%%%%%
% Pages consacrées aux différentes tables de référencement
\pagebreak 
\phantomsection
\addstarredchapter{Table des figures}
\listoffigures 
\pagebreak
\phantomsection
\addstarredchapter{Table des tableaux}
\listoftables
\pagebreak 
\phantomsection
\addstarredchapter{Table des algorithmes}
\listofalgorithms
\pagebreak 
\phantomsection
\addstarredchapter{Table des équations}
\listofmyequations

%%%%%%%%%%%%%%%%%%%%%%%%%%%%%%%%%%%%%%%%%%%%%%%%%%%%%%%%%%%%
% Insertion des annexes
\begin{appendix}
% @Author: Jérémy Coatelen
% @Date:   2015-05-05 20:55:05
% @Last Modified by:   dash00
% @Last Modified time: 2015-05-05 21:05:21

\chapter{Une chose intéressante}
\label{ann:1}


\chapter{Une autre chose intéressante}
\label{ann:2}

\end{appendix}

%%%%%%%%%%%%%%%%%%%%%%%%%%%%%%%%%%%%%%%%%%%%%%%%%%%%%%%%%%%%
% Pages de bibliographie
% @Author: Jérémy Coatelen
% @Date:   2015-05-05 20:54:17
% @Last Modified by:   dash00
% @Last Modified time: 2015-05-05 21:05:32

\clearpage
\phantomsection
\addcontentsline{toc}{chapter}{Bibliographie de l'auteur}
\chapter*{Bibliographie de l'auteur}
\label{biblioauteur}


% Ici la biblio de l'auteur



% ===========================================================================


\clearpage
\phantomsection
\addcontentsline{toc}{chapter}{Bibliographie complète}
\bibliographystyle{\mybibliostyle}
\bibliography{mybib}

%%%%%%%%%%%%%%%%%%%%%%%%%%%%%%%%%%%%%%%%%%%%%%%%%%%%%%%%%%%%
% Nouvelle insertion du résumé
% @Author: Jérémy Coatelen
% @Date:   2015-05-05 20:21:23
% @Last Modified by:   dash00
% @Last Modified time: 2015-05-05 21:05:05

\clearpage
\phantomsection
%\addstarredchapter{Résumé}

\markboth{Abstract}{Résumé}

\vspace{4cm}

\begin{center}
\begin{large}\textbf{Titre}\end{large}\\
\end{center}
\vspace{1cm}

\begin{center}
\begin{large}\textbf{Résumé}\end{large}\\
\end{center}
\vspace{1cm}


Mots-clés : 


\vspace{2.5cm}
\ligne
\vspace{2.5cm}

\begin{center}
\begin{large}\textbf{Title}\end{large}\\
\end{center}
\vspace{1cm}



\vspace{1cm}
\begin{center}
\begin{large}\textbf{Abstract}\end{large}\\
\end{center}
\vspace{1cm}


Keywords : 



\end{document}
